%%%%%%%%%%%%%%%%%%%%%%%%%%%%%%%%%%%%%%%%%%%%%%%%%%%%%%%%%%%%%%%%%%%%
%% I, the copyright holder of this work, release this work into the
%% public domain. This applies worldwide. In some countries this may
%% not be legally possible; if so: I grant anyone the right to use
%% this work for any purpose, without any conditions, unless such
%% conditions are required by law.
%%%%%%%%%%%%%%%%%%%%%%%%%%%%%%%%%%%%%%%%%%%%%%%%%%%%%%%%%%%%%%%%%%%%

\documentclass{beamer}
\usetheme[logo=autobotz_logo.png]{fibeamer}
\usepackage[utf8]{inputenc}
\usepackage[
  english, %% By using `czech` or `slovak` as the main locale
                %% instead of `english`, you can typeset the
                %% presentation in either Czech or Slovak,
                %% respectively.
  czech, slovak, main=brazilian %% The additional keys allow foreign texts to be
]{babel}        %% typeset as follows:
%%
%%   \begin{otherlanguage}{czech}   ... \end{otherlanguage}
%%   \begin{otherlanguage}{slovak}  ... \end{otherlanguage}
%%
%% These macros specify information about the presentation
\title{Título do projeto} %% that will be typeset on the
\subtitle{Subtítulo} %% title page.
\author{Autoras(es)}
%% These additional packages are used within the document:
\usepackage{ragged2e}  % `\justifying` text
\usepackage{booktabs}  % Tables
\usepackage{tabularx}
\usepackage{tikz}      % Diagrams
\usetikzlibrary{calc, shapes, backgrounds}
\usepackage{amsmath, amssymb}
\usepackage{url}       % `\url`s
\usepackage{listings}  % Code listings
\frenchspacing

\begin{document}
\begin{otherlanguage}{brazilian}
  \frame{\maketitle}

  \AtBeginSection[]{% Print an outline at the beginning of sections
    \begin{frame}<beamer>
      \frametitle{Índice da \thesectionª sessão}
      \tableofcontents[currentsection]
    \end{frame}}
\begin{darkframes}

  \section{1.0 Título}
  
	\subsection{1.1 Título}
	  \begin{frame}{Título do slide}
      \framesubtitle{Subtítulo}
        \begin{columns}[onlytextwidth]
          \column{\textwidth}
            \begin{itemize}
            \item lista
            \end{itemize}
        \end{columns}  
      \end{frame}
      
      \subsection{2.2 Título}
        \begin{frame}{Título do slide}
      	  \framesubtitle{Subtítulo}

       \section{2.0 Título}

  \end{darkframes}

    
\end{otherlanguage}
\end{document}
